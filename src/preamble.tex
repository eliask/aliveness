% Preamble for The Four Axiomatic Dilemmas
% Professional book typesetting with XeLaTeX

\documentclass[11pt,twoside,openright]{book}

% ============================================================================
% PAGE GEOMETRY
% ============================================================================
\usepackage[
    paperwidth=6in,
    paperheight=9in,
    top=0.75in,
    bottom=0.75in,
    inner=0.85in,  % Larger margin for binding
    outer=0.6in,
    headheight=24pt,  % Increased for fancy headers
    includeheadfoot
]{geometry}

% ============================================================================
% FONTS (XeLaTeX)
% ============================================================================
\usepackage{fontspec}

% Professional fonts for optimal readability
% Install with: brew tap homebrew/cask-fonts && brew install font-libertinus
% Fallback to Times New Roman if Libertinus not available
\IfFontExistsTF{Libertinus Serif}{
    \setmainfont{Libertinus Serif}[
        Ligatures=TeX,
        Numbers=OldStyle,
        Scale=1.0
    ]
    \setsansfont{Libertinus Sans}
    \setmonofont{Libertinus Mono}[Scale=0.9]
}{
    % Fallback to system fonts
    \setmainfont{Times New Roman}[Ligatures=TeX]
    \setsansfont{Helvetica}
    \setmonofont{Courier}[Scale=0.9]
}

% ============================================================================
% TYPOGRAPHY
% ============================================================================
\usepackage{microtype}       % Better typography
\usepackage{setspace}        % Line spacing control
\setstretch{1.15}            % 1.15 line spacing

% Paragraph formatting
\setlength{\parindent}{1em}
\setlength{\parskip}{0pt}

% Prevent overfull hboxes
\sloppy
\hyphenpenalty=5000
\tolerance=1000

% ============================================================================
% MATH (must load before cleveref)
% ============================================================================
\usepackage{amsmath}
\usepackage{amssymb}

% ============================================================================
% CROSS-REFERENCES AND HYPERLINKS
% ============================================================================
\usepackage{hyperref}
\hypersetup{
    colorlinks=true,
    linkcolor=blue,
    urlcolor=blue,
    citecolor=blue,
    bookmarksdepth=2,
    pdfauthor={Elias Kunnas},
    pdftitle={The Four Axiomatic Dilemmas},
    pdfsubject={Principles of Telic Systems},
    pdfkeywords={Aliveness, Wonder, SORT Framework, Civilization, AI Alignment, Human alignment, Telic systems}
}

% Enhanced cross-referencing (must come after hyperref)
\usepackage{cleveref}

% Configure cleveref labels
\crefname{section}{Section}{Sections}
\crefname{subsection}{Section}{Sections}
\crefname{subsubsection}{Section}{Sections}
\crefname{chapter}{Chapter}{Chapters}
\crefname{part}{Part}{Parts}
\crefname{equation}{Equation}{Equations}
\crefname{figure}{Figure}{Figures}
\crefname{table}{Table}{Tables}
\crefname{appendix}{Appendix}{Appendices}

% Note: After \appendix command, chapters become appendices
% Use \appendixname to reference them as "Appendix 1" etc.

% ============================================================================
% TABLE OF CONTENTS
% ============================================================================
\setcounter{tocdepth}{2}     % Show sections in TOC (Part=0, Chapter=1, Section=2)
\setcounter{secnumdepth}{3}  % Number subsubsections (keeps numbering in text)

% Fix appendix section numbering to show A.1, B.1, etc.
\makeatletter
\@addtoreset{section}{chapter}     % Reset section counter for each chapter
\@addtoreset{subsection}{chapter}  % Reset subsection counter for each chapter (appendices use subsections directly)
\makeatother

% ============================================================================
% HEADERS AND FOOTERS
% ============================================================================
\usepackage{fancyhdr}
\pagestyle{fancy}
\fancyhf{}  % Clear all headers/footers

% Enhanced headers: Part context on left, section on right
\fancyhead[LE]{\small\itshape\nouppercase{\partname\ \thepart\ |\ \leftmark}}
\fancyhead[RO]{\small\itshape\nouppercase{\rightmark}}

% Page numbers
\fancyfoot[LE,RO]{\small\thepage}

% Plain style for chapter opening pages
\fancypagestyle{plain}{
    \fancyhf{}
    \fancyfoot[C]{\small\thepage}
    \renewcommand{\headrulewidth}{0pt}
}

% *** FIX STARTS HERE ***
\usepackage{iftex} % Add this package to detect the engine

% Define dummy commands FIRST for Pandoc (before the main definitions)
\ifdefined\pandocVersion
    % Pandoc is running - define no-op commands early
    \providecommand{\startNarrativeChapter}{}
    \providecommand{\stopNarrativeChapter}{}
    \providecommand{\setDefaultNumbering}{}
    \providecommand{\setNarrativeNumbering}{}
\fi

% Check if we're being processed by Pandoc
\ifdefined\pandocVersion
    % Pandoc is running - skip all titlesec commands
\else
    % XeLaTeX or other processor - include titlesec
    % ============================================================================
    % CHAPTER AND SECTION FORMATTING
    % ============================================================================
    \usepackage{titlesec}

    % Part formatting (for PART I, II, III, etc.)
    \titleformat{\part}[display]
        {\centering\Huge\bfseries}
        {\MakeUppercase{\partname} \thepart}
        {20pt}
        {\Huge}

    % Chapter formatting
    \titleformat{\chapter}[display]
        {\normalfont\huge\bfseries}
        {\chaptertitlename\ \thechapter}
        {20pt}
        {\Huge}

    % Section formatting
    \titleformat{\section}
        {\normalfont\Large\bfseries}
        {\thesection}
        {1em}
        {}

    \titleformat{\subsection}
        {\normalfont\large\bfseries}
        {\thesubsection}
        {1em}
        {}

    \titleformat{\subsubsection}
        {\normalfont\normalsize\bfseries}
        {\thesubsubsection}
        {1em}
        {}

    % Improved spacing around sections
    \titlespacing{\section}{0pt}{2.5ex plus 1ex minus .2ex}{1.5ex plus .2ex}
    \titlespacing{\subsection}{0pt}{2ex plus 1ex minus .2ex}{1ex plus .2ex}
    \titlespacing{\subsubsection}{0pt}{1.5ex plus .5ex minus .2ex}{0.8ex plus .2ex}

    % ============================================================================
    % DUAL NUMBERING SYSTEM (Gnostic/Mythos Integration)
    % ============================================================================
    % This allows switching between default numeric and narrative Roman styles
    % without breaking cross-references or ToC, per the Gemini-Claude synthesis.

    % --- Default Numeric Style (Restored with \endNarrativeChapter) ---
    \providecommand{\setDefaultNumbering}{}
    \renewcommand{\setDefaultNumbering}{%
      \titleformat{\section}
        {\Large\bfseries}
        {\thesection} % Displays as chapter.section (e.g., "4.1")
        {1em}
        {}
      \titleformat{\subsection}
        {\large\bfseries}
        {\thesubsection} % Displays as chapter.section.subsection
        {1em}
        {}
    }

    % --- Narrative Numeric Style (Activated with \startNarrativeChapter) ---
    \providecommand{\setNarrativeNumbering}{}
    \renewcommand{\setNarrativeNumbering}{%
      \titleformat{\section}
        {\Large\bfseries}
        {\thechapter.\arabic{section}} % Displays as numeric (e.g., "13.1")
        {1em}
        {}
      \titleformat{\subsection}
        {\large\bfseries}
        {\thechapter.\arabic{section}.\arabic{subsection}} % ToC/ref friendly: 13.2.3
        {1em}
        {}
    }

    % Define the user-facing commands
    \providecommand{\startNarrativeChapter}{}
    \renewcommand{\startNarrativeChapter}{\setNarrativeNumbering}
    \providecommand{\stopNarrativeChapter}{}
    \renewcommand{\stopNarrativeChapter}{\setDefaultNumbering}

    % Ensure the default is set at the beginning of the document
    \AtBeginDocument{\setDefaultNumbering}
\fi
% *** FIX ENDS HERE ***


% ============================================================================
% SPECIAL ENVIRONMENTS
% ============================================================================

% For block quotes
\usepackage{mdframed}

% Block quotes - enhanced styling
\usepackage{csquotes}
\renewenvironment{quote}
  {\list{}{\leftmargin=2em\rightmargin=0em\itshape}\item[]}
  {\endlist}

% ============================================================================
% LISTS
% ============================================================================
\usepackage{enumitem}
\setlist{nosep, leftmargin=*, labelindent=\parindent, itemsep=0.5ex}

% ============================================================================
% EPISTEMIC STATUS BOXES
% ============================================================================
\usepackage{xcolor}
\usepackage{tcolorbox}
\tcbuselibrary{skins,breakable}

% Define epistemic status colors
\definecolor{epistemic-high}{RGB}{34, 139, 34}      % Forest green
\definecolor{epistemic-medium}{RGB}{255, 165, 0}    % Orange
\definecolor{epistemic-low}{RGB}{178, 34, 34}       % Firebrick red
\definecolor{epistemic-mixed}{RGB}{218, 165, 32}    % Goldenrod

% Epistemic status box environment
\newtcolorbox{epistemicbox}[2][]{
  colback=#2!5!white,
  colframe=#2!75!black,
  fonttitle=\bfseries,
  coltitle=black,
  title=Epistemic Status: #1,
  breakable,
  enhanced,
  attach boxed title to top left={yshift=-2mm, xshift=3mm},
  boxed title style={colback=#2!75!black, colframe=#2!75!black},
  top=3mm,
  bottom=3mm,
  left=3mm,
  right=3mm
}

% Convenience commands for each tier
\newcommand{\epistemichigh}[2]{%
  \begin{epistemicbox}[#1]{epistemic-high}
  #2
  \end{epistemicbox}
}

\newcommand{\epistemicmedium}[2]{%
  \begin{epistemicbox}[#1]{epistemic-medium}
  #2
  \end{epistemicbox}
}

\newcommand{\epistemiclow}[2]{%
  \begin{epistemicbox}[#1]{epistemic-low}
  #2
  \end{epistemicbox}
}

\newcommand{\epistemicmixed}[2]{%
  \begin{epistemicbox}[#1]{epistemic-mixed}
  #2
  \end{epistemicbox}
}

% SORT archetype notation - monospace for alignment
\newcommand{\sortarch}[4]{%
  \texttt{[S#1\,O#2\,R#3\,T#4]}%
}

% Pandoc-generated command for tight lists
\providecommand{\tightlist}{%
  \setlength{\itemsep}{0pt}\setlength{\parskip}{0pt}}

% ============================================================================
% TABLES AND FIGURES
% ============================================================================
\usepackage{booktabs}   % Professional tables
\usepackage{longtable}  % Multi-page tables
\usepackage{tabularx}   % Auto-width tables
\usepackage{multirow}   % Multi-row cells in tables
\usepackage{graphicx}   % For figures if needed
\usepackage{float}      % For H float option

% TikZ for high-quality diagrams
\usepackage{tikz}
\usetikzlibrary{arrows.meta,positioning,shapes.geometric,calc,decorations.pathreplacing}

% ============================================================================
% CODE BLOCKS (for ASCII art diagrams)
% ============================================================================
\usepackage{fancyvrb}
\usepackage{listings}

% Verbatim environment that doesn't break pages
\usepackage{needspace}

% Custom verbatim environment for diagrams
\DefineVerbatimEnvironment{diagram}{Verbatim}{
    fontsize=\small,
    fontfamily=tt,
    samepage=true
}

% ============================================================================
% ACRONYM MANAGEMENT
% ============================================================================
\usepackage{acronym}

% ============================================================================
% INDEX GENERATION
% ============================================================================
\usepackage{makeidx}
\makeindex

% ============================================================================
% MISCELLANEOUS
% ============================================================================

% Better spacing in arrays and matrices
\usepackage{array}

% For strikethrough text
\usepackage[normalem]{ulem}

% For colored text
\usepackage{xcolor}

% ============================================================================
% VISUAL HIERARCHY BOXES
% ============================================================================
\usepackage{tcolorbox}
\tcbuselibrary{skins,breakable}

% Key principle boxes (for major theorems and laws)
\newtcolorbox{keyprinciple}[1][]{
    colback=blue!5!white,
    colframe=blue!75!black,
    fonttitle=\bfseries,
    title=#1,
    breakable,
    enhanced jigsaw,
    before skip=10pt,
    after skip=10pt
}

% Definition boxes
\newtcolorbox{definition}[1][]{
    colback=gray!5!white,
    colframe=gray!75!black,
    fonttitle=\bfseries,
    title=Definition: #1,
    breakable,
    before skip=8pt,
    after skip=8pt
}

% Example boxes
\newtcolorbox{example}[1][]{
    colback=green!5!white,
    colframe=green!60!black,
    fonttitle=\bfseries,
    title=Example: #1,
    breakable,
    before skip=8pt,
    after skip=8pt
}

% Prevent widows and orphans
\widowpenalty=10000
\clubpenalty=10000
\raggedbottom

% Allow slightly more space between words if needed
\emergencystretch=3em

% ============================================================================
% CUSTOM COMMANDS
% ============================================================================

% Note: \Omega and \Psi are already defined by LaTeX as math symbols
% We don't need to redefine them - they work as-is in math mode
% Just use $\Omega$ and $\Psi$ directly in the text

% SORT notation
\newcommand{\SORT}{\textsc{sort}}
\newcommand{\SORTVC}{\textsc{sortvc}}
\newcommand{\pSORT}{\textsc{psort}}
\newcommand{\tSORT}{\textsc{tsort}}
\newcommand{\bioSORT}{\textsc{bio-sort}}

% States
\newcommand{\ALPHA}{\textsc{alpha}}
\newcommand{\BETA}{\textsc{beta}}
\newcommand{\GAMMA}{\textsc{gamma}}
\newcommand{\ENTROPIC}{\textsc{entropic}}

% Virtues (IFHS)
\newcommand{\IFHS}{\textsc{ifhs}}

% ============================================================================
% MATHEMATICAL NOTATION SHORTCUTS (Corrected & Harmonized)
% ============================================================================

% Use standard LaTeX math commands for consistency and portability.
% The \ensuremath command ensures they work correctly in and out of math mode.

% Civilizational Dynamics
\newcommand{\Omegac}{\ensuremath{\Omega}}           % State Coherence
\newcommand{\Alphac}{\ensuremath{\mathrm{A}}}       % Action Vector (Use Roman 'A' for distinction)
\newcommand{\Psic}{\ensuremath{\Psi}}               % Telic Potential (Soul)
\newcommand{\Nuc}{\ensuremath{\mathrm{N}}}          % Gnomic Potential (Mind) (Use Roman 'N')
\newcommand{\Kappac}{\ensuremath{\mathrm{K}}}       % Action Potential (Use Roman 'K')
\newcommand{\sigmac}{\ensuremath{\sigma_A}}         % Axiological variance

% Backward compatibility aliases (for standalone symbol use)
\newcommand{\Alpha}{\mathrm{A}}                     % Action Vector symbol (use in math mode)
\newcommand{\Nu}{\mathrm{N}}                        % Gnomic symbol (use in math mode)
\newcommand{\Kappa}{\mathrm{K}}                     % Action Potential symbol (use in math mode)

% Personal Dynamics (subscript p)
\newcommand{\Omegap}{\ensuremath{\Omega_p}}         % Personal Coherence
\newcommand{\Alphap}{\ensuremath{\mathrm{A}_p}}     % Personal Action

% Subscripted Omega Variants
\newcommand{\Omegainst}{\ensuremath{\Omega_{\text{institutional}}}}
\newcommand{\Omegakin}{\ensuremath{\Omega_{\text{kinetic}}}}
\newcommand{\Omegapop}{\ensuremath{\Omega_{\text{population}}}}

% Em dash
\newcommand{\emdash}{---}

% Status symbols (for tables and lists)
\newcommand{\pass}{\textcolor{green!70!black}{$\checkmark$}}
\newcommand{\fail}{\textcolor{red!70!black}{$\times$}}

% ============================================================================
% END PREAMBLE
% ============================================================================

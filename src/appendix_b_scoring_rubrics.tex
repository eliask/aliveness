\chapter{SORT Scoring Rubrics}
\label{app:scoring-rubrics}

\subsection{Purpose and Scope}\label{purpose-and-scope}

\Cref{app:lexicon} defines what the SORT axes \textbf{mean}. This appendix defines \textbf{how to measure them}.

These rubrics provide operational protocols for scoring civilizations on the SORT framework. They transform abstract concepts (Sovereignty, Organization, Reality, Telos) into measurable indicators based on observable evidence.

\textbf{Critical caveat:} These rubrics represent \textbf{Version 1.0}---initial operationalization based on theoretical reasoning and preliminary case study work. They require empirical validation, inter-rater reliability testing, and iterative refinement. Use them as starting points, not gospel.

\textbf{Target audience:}
\begin{itemize}
\item Researchers conducting systematic historical analysis
\item Validators testing the framework's predictions
\item Anyone attempting to apply SORT to real civilizations
\end{itemize}

\subsection{I. General Methodology}\label{i.-general-methodology}

\textbf{Scoring Process:} Identify polity and time period precisely; gather multi-source evidence; apply rubrics below; triangulate indicators; document reasoning; flag uncertainty.

\textbf{Scoring Scale:} -1.0 to +1.0 in 0.1 increments (V ranges 0-10). Extreme scores (±1.0) require overwhelming evidence; most civilizations cluster in ±0.7 range.

\textbf{Evidence Quality:} Primary sources (legal codes, census, economic data) > Archaeological evidence > Contemporary accounts > Secondary scholarship > Inference (weakest).

\subsection{II. S-Axis Scoring Rubric (Sovereignty)}\label{ii.-s-axis-scoring-rubric-sovereignty}

\textbf{Core Question:} Who is the fundamental unit of moral and political concern?

\textbf{-1.0 (Pure Individual Sovereignty):}
\begin{itemize}
\item \textbf{Legal system:} Rights vest in individuals. No collective privileges or obligations.
\item \textbf{Political structure:} One person, one vote. No ethnic/tribal quotas or preferences.
\item \textbf{Cultural norms:} Individualism glorified. ``Self-made man'' ideal. Weak kinship bonds.
\item \textbf{Economic system:} Private property absolute. Minimal wealth redistribution.
\item \textbf{Example:} Classical Liberal ideal (theoretical; rare in pure form)
\end{itemize}

\textbf{-0.7 (Strong Individual Bias):}
\begin{itemize}
\item \textbf{Legal:} Individual rights strongly protected but some collective obligations exist (e.g., conscription, taxation).
\item \textbf{Political:} Democratic with strong individual liberties. Weak group identity politics.
\item \textbf{Cultural:} ``You can be anything'' ethos. Weak ethnic/religious identity.
\item \textbf{Economic:} Market capitalism with limited redistribution.
\item \textbf{Examples:} USA 1800-1960, Victorian Britain, early Roman Republic
\end{itemize}

\textbf{-0.3 (Mild Individual Bias):}
\begin{itemize}
\item \textbf{Legal:} Individual and collective rights balanced. Some group-based policies (affirmative action).
\item \textbf{Political:} Democracy with identity politics emerging.
\item \textbf{Cultural:} Individualism valued but group identities gaining salience.
\item \textbf{Economic:} Mixed economy with moderate redistribution.
\item \textbf{Examples:} Modern USA 1970-2010, Modern Europe
\end{itemize}

\textbf{0.0 (Balanced/Contested):}
\begin{itemize}
\item \textbf{Legal:} Hybrid system with individual and collective rights in tension.
\item \textbf{Political:} Factional politics organized around group identities.
\item \textbf{Cultural:} Culture war between individualists and collectivists.
\item \textbf{Examples:} Transition states, civil wars, contested ideologies
\end{itemize}

\textbf{+0.3 (Mild Collective Bias):}
\begin{itemize}
\item \textbf{Legal:} Group rights recognized. Some laws favor collective over individual.
\item \textbf{Political:} Democracy with explicit ethnic/religious representation.
\item \textbf{Cultural:} ``Duties to community'' emphasized. Strong in-group loyalty.
\item \textbf{Economic:} Significant redistribution; ``social solidarity'' norms.
\item \textbf{Examples:} Japan, Israel, Singaporean meritocracy
\end{itemize}

\textbf{+0.7 (Strong Collective Bias):}
\begin{itemize}
\item \textbf{Legal:} Collective (family, tribe, nation) is primary legal unit. Individual subordinate.
\item \textbf{Political:} Representation by group identity (ethnic, religious). Tribal federalism.
\item \textbf{Cultural:} Shame culture. Honor of group \textgreater{} individual desire. Strong kinship networks.
\item \textbf{Economic:} Collective ownership or strong redistribution within group.
\item \textbf{Examples:} Pre-modern tribes, Imperial Japan, Zionist Israel, Classical Sparta
\end{itemize}

\textbf{+1.0 (Pure Collective Sovereignty):}
\begin{itemize}
\item \textbf{Legal:} No individual rights. All obligations to collective.
\item \textbf{Political:} Individuals exist to serve state/nation/tribe.
\item \textbf{Cultural:} Total subordination of individual to collective. Extreme shame culture.
\item \textbf{Economic:} Communal property. No private accumulation.
\item \textbf{Example:} Stalinist USSR, Khmer Rouge, theoretical fascist/communist ideals
\end{itemize}

\textbf{Key Indicators for S-Axis:}

{\small
\begin{longtable}[]{@{}p{2.8cm}p{4.2cm}p{4.2cm}@{}}
\toprule
\textbf{Indicator} & \textbf{Individual (-)} & \textbf{Collective (+)} \\
\midrule
\endhead
\bottomrule
\endlastfoot
\textbf{Legal:} Property rights & Absolute private property & Communal/state ownership \\
\textbf{Legal:} Criminal justice & Individual guilt/innocence & Collective punishment (blood feuds, guilt by association) \\
\textbf{Political:} Representation & One person, one vote & Tribal/ethnic/religious quotas \\
\textbf{Cultural:} Marriage norms & Individual choice & Arranged by family/group \\
\textbf{Cultural:} Career choice & Individual meritocracy & Determined by caste/family/group need \\
\textbf{Cultural:} Honor vs.~Dignity & Dignity culture (individual worth) & Honor culture (group reputation) \\
\textbf{Economic:} Taxation & Low, voluntary if possible & High, compulsory for redistribution \\
\textbf{Social:} In-group nepotism & Stigmatized & Normalized and expected \\
\end{longtable}
}

\subsection{III. O-Axis Scoring Rubric (Organization)}\label{iii.-o-axis-scoring-rubric-organization}

\textbf{Core Question:} How is order created and maintained?

\textbf{-1.0 (Pure Emergence):}
\begin{itemize}
\item \textbf{Legal:} Common law evolved from precedent. No codification.
\item \textbf{Economic:} Pure laissez-faire. No central planning.
\item \textbf{Political:} Minimal government. Spontaneous order (markets, norms).
\item \textbf{Infrastructure:} Organic city growth. No urban planning.
\item \textbf{Example:} Medieval merchant cities, early Anglo-Saxon law
\end{itemize}

\textbf{-0.7 (Strong Emergence Bias):}
\begin{itemize}
\item \textbf{Legal:} Common law primary, some statutory law.
\item \textbf{Economic:} Market economy with minimal regulation.
\item \textbf{Political:} Limited government, federalism, decentralization.
\item \textbf{Infrastructure:} Mostly organic growth with light zoning.
\item \textbf{Examples:} USA 1800-1900, Hong Kong, medieval Venice
\end{itemize}

\textbf{-0.3 (Mild Emergence Bias):}
\begin{itemize}
\item \textbf{Legal:} Mixed common law and statutory law.
\item \textbf{Economic:} Market economy with regulatory framework.
\item \textbf{Political:} Democracy with checks and balances, devolution.
\item \textbf{Infrastructure:} Mix of organic and planned development.
\item \textbf{Examples:} Modern UK, Canada, Australia
\end{itemize}

\textbf{0.0 (Balanced/Contested):}
\begin{itemize}
\item \textbf{Legal:} Hybrid systems in active tension.
\item \textbf{Economic:} Mixed economy with planning and markets competing.
\item \textbf{Political:} Centralization vs.~federalism debates intense.
\item \textbf{Examples:} Transition states, constitutional crises
\end{itemize}

\textbf{+0.3 (Mild Design Bias):}
\begin{itemize}
\item \textbf{Legal:} Civil law codes with some case law.
\item \textbf{Economic:} Regulated capitalism with active industrial policy.
\item \textbf{Political:} Strong central government with regional administration.
\item \textbf{Infrastructure:} Mostly planned cities with some organic elements.
\item \textbf{Examples:} Modern France, Germany, Japan
\end{itemize}

\textbf{+0.7 (Strong Design Bias):}
\begin{itemize}
\item \textbf{Legal:} Comprehensive civil law codes. Top-down legislation.
\item \textbf{Economic:} Command economy or heavily managed capitalism.
\item \textbf{Political:} Centralized state with weak local autonomy.
\item \textbf{Infrastructure:} Fully planned cities (grid patterns, zoning).
\item \textbf{Examples:} Soviet Union, Napoleonic France, Qin Dynasty China, Singapore
\end{itemize}

\textbf{+1.0 (Pure Design):}
\begin{itemize}
\item \textbf{Legal:} Complete codification. No precedent or emergence.
\item \textbf{Economic:} Total central planning. No markets.
\item \textbf{Political:} Totalitarian central control.
\item \textbf{Infrastructure:} All development centrally planned and executed.
\item \textbf{Example:} North Korea, theoretical total planning state
\end{itemize}

\textbf{Key Indicators for O-Axis:}

{\small
\begin{longtable}[]{@{}p{2.8cm}p{4.2cm}p{4.2cm}@{}}
\toprule
\textbf{Indicator} & \textbf{Emergence (-)} & \textbf{Design (+)} \\
\midrule
\endhead
\bottomrule
\endlastfoot
\textbf{Legal:} Law source & Precedent, custom, evolved & Codified, legislated, imposed \\
\textbf{Economic:} Market structure & Free markets, price discovery & Central planning, price controls \\
\textbf{Political:} Power distribution & Federalism, subsidiarity & Centralization, top-down \\
\textbf{Urban:} City planning & Organic growth & Master-planned (grid cities) \\
\textbf{Cultural:} Language policy & Natural evolution, dialects & Standardized, Academie-style control \\
\textbf{Infrastructure:} Development & Bottom-up, entrepreneur-led & State-led, Five-Year Plans \\
\end{longtable}
}

\subsection{IV. R-Axis Scoring Rubric (Reality)}\label{iv.-r-axis-scoring-rubric-reality}

\textbf{Core Question:} What is the ultimate source of truth and authority?

\textbf{-1.0 (Pure Mythos):}
\begin{itemize}
\item \textbf{Epistemology:} Revelation, tradition, sacred texts are only truth sources.
\item \textbf{Decision-making:} Oracles, priests, tradition exclusively consulted.
\item \textbf{Education:} Religious instruction only. No empirical science.
\item \textbf{Discourse:} Heresy prosecuted. Dogma enforced.
\item \textbf{Example:} Medieval theocracy, Taliban Afghanistan
\end{itemize}

\textbf{-0.7 (Strong Mythos Bias):}
\begin{itemize}
\item \textbf{Epistemology:} Religious/traditional authority dominant; science subordinate.
\item \textbf{Decision-making:} Religious leaders have veto power over policy.
\item \textbf{Education:} Religious instruction primary; some practical skills.
\item \textbf{Discourse:} Blasphemy laws enforced. Orthodoxy protected.
\item \textbf{Examples:} Medieval Christendom, Safavid Persia, modern Iran
\end{itemize}

\textbf{-0.3 (Mild Mythos Bias):}
\begin{itemize}
\item \textbf{Epistemology:} Both tradition and empiricism valued; tradition given priority in conflicts.
\item \textbf{Decision-making:} Traditional values shape policy more than data.
\item \textbf{Education:} Liberal arts + religion emphasized over STEM.
\item \textbf{Discourse:} Political correctness, ``sacred values'' limit inquiry.
\item \textbf{Examples:} Modern social democracies with strong ideological taboos
\end{itemize}

\textbf{0.0 (Balanced/Contested):}
\begin{itemize}
\item \textbf{Epistemology:} Culture war between Mythos and Gnosis factions.
\item \textbf{Decision-making:} Policy battles between ideologues and technocrats.
\item \textbf{Education:} Humanities vs.~STEM funding battles.
\item \textbf{Examples:} Culture war states, USA 2010s-2020s
\end{itemize}

\textbf{+0.3 (Mild Gnosis Bias):}
\begin{itemize}
\item \textbf{Epistemology:} Empiricism dominant but traditional wisdom still respected.
\item \textbf{Decision-making:} Technocrats influential; some deference to tradition.
\item \textbf{Education:} STEM prioritized but humanities still valued.
\item \textbf{Discourse:} Free inquiry norm but some sacred cows remain.
\item \textbf{Examples:} Postwar USA, modern Germany
\end{itemize}

\textbf{+0.7 (Strong Gnosis Bias):}
\begin{itemize}
\item \textbf{Epistemology:} Empiricism, rationalism dominant. Tradition dismissed.
\item \textbf{Decision-making:} Technocracy. Data-driven policy. ``Science says\ldots{}''
\item \textbf{Education:} STEM-focused. Humanities declining. Utilitarian.
\item \textbf{Discourse:} Rationalism valorized. Religion marginalized.
\item \textbf{Examples:} Soviet scientific materialism, Singapore technocracy, modern China
\end{itemize}

\textbf{+1.0 (Pure Gnosis):}
\begin{itemize}
\item \textbf{Epistemology:} Only empirical science accepted. All tradition rejected.
\item \textbf{Decision-making:} Pure technocracy. Algorithm-driven governance.
\item \textbf{Education:} Pure STEM. No arts, no humanities, no philosophy.
\item \textbf{Discourse:} Logical positivism enforced. All non-falsifiable claims banned.
\item \textbf{Example:} Theoretical hyper-rationalist state (rare in practice)
\end{itemize}

\textbf{Key Indicators for R-Axis:}

{\small
\begin{longtable}[]{@{}p{2.8cm}p{4.2cm}p{4.2cm}@{}}
\toprule
\textbf{Indicator} & \textbf{Mythos (-)} & \textbf{Gnosis (+)} \\
\midrule
\endhead
\bottomrule
\endlastfoot
\textbf{Epistemology:} Authority source & Sacred texts, tradition, elders & Data, experiments, peer review \\
\textbf{Policy:} Basis for decisions & Values, ideology, tradition & Statistics, cost-benefit analysis \\
\textbf{Education:} Curriculum focus & Humanities, religion, classics & STEM, engineering, applied sciences \\
\textbf{Legal:} Basis for law & Divine command, natural law & Pragmatic utility, evidence \\
\textbf{Cultural:} Attitude to science & Subordinate to religion/tradition & Supreme arbiter of truth \\
\textbf{Discourse:} Limits on speech & Heresy, blasphemy prosecuted & Empirical falsity prosecuted \\
\end{longtable}
}

\subsection{V. T-Axis Scoring Rubric (Telos)}\label{v.-t-axis-scoring-rubric-telos}

\textbf{Core Question:} What is the civilization's ultimate purpose?

\textbf{-1.0 (Pure Homeostasis):}
\begin{itemize}
\item \textbf{Rhetoric:} ``Safety,'' ``stability,'' ``sustainability,'' ``preservation'' dominate discourse.
\item \textbf{Policy:} All change resisted. Zero-risk mentality. Precautionary principle absolute.
\item \textbf{Culture:} Risk-taking stigmatized. Nostalgia for past golden age.
\item \textbf{Economics:} Zero-growth economy. Degrowth movement.
\item \textbf{Demographics:} Below-replacement fertility accepted/celebrated.
\item \textbf{Example:} Hospice civilization end-state, theoretical stagnation
\end{itemize}

\textbf{-0.7 (Strong Homeostasis Bias):}
\begin{itemize}
\item \textbf{Rhetoric:} ``Protect what we have.'' Change as threat.
\item \textbf{Policy:} Heavy regulation, risk aversion, status quo bias.
\item \textbf{Culture:} Comfort, safety, therapy prioritized over achievement.
\item \textbf{Economics:} Redistribution \textgreater{} growth. Welfare state expansion.
\item \textbf{Demographics:} TFR 1.2-1.4. Declining population accepted.
\item \textbf{Examples:} Modern Western Europe, Japan, modern USA post-2010
\end{itemize}

\textbf{-0.3 (Mild Homeostasis Bias):}
\begin{itemize}
\item \textbf{Rhetoric:} ``Sustainable growth.'' Change with caution.
\item \textbf{Policy:} Moderate regulation. Risk management (not elimination).
\item \textbf{Culture:} Balance between comfort and achievement.
\item \textbf{Economics:} Mixed economy. Growth + redistribution.
\item \textbf{Demographics:} TFR 1.5-1.8. Demographic decline acknowledged as problem.
\item \textbf{Examples:} USA 1970-2000, modern Canada
\end{itemize}

\textbf{0.0 (Balanced/Contested):}
\begin{itemize}
\item \textbf{Rhetoric:} Intense cultural conflict over growth vs.~sustainability.
\item \textbf{Policy:} Policy whiplash between growth and precaution factions.
\item \textbf{Culture:} Generational warfare (old=homeostasis, young=metamorphosis or vice versa).
\item \textbf{Examples:} Transition states, contested elections, reform periods
\end{itemize}

\textbf{+0.3 (Mild Metamorphosis Bias):}
\begin{itemize}
\item \textbf{Rhetoric:} ``Progress,'' ``development,'' ``innovation'' valued.
\item \textbf{Policy:} Pro-growth policies. Calculated risk-taking.
\item \textbf{Culture:} Achievement, striving, ambition encouraged.
\item \textbf{Economics:} Growth prioritized. Investment \textgreater{} consumption.
\item \textbf{Demographics:} TFR 1.9-2.2. Replacement or slight growth.
\item \textbf{Examples:} Postwar USA 1945-1970, modern South Korea
\end{itemize}

\textbf{+0.7 (Strong Metamorphosis Bias):}
\begin{itemize}
\item \textbf{Rhetoric:} ``Conquest,'' ``glory,'' ``empire,'' ``transcendence'' dominate.
\item \textbf{Policy:} Aggressive expansion (geographic, economic, technological).
\item \textbf{Culture:} Heroism, sacrifice for future glorified. Spartan ethic.
\item \textbf{Economics:} High investment. Capital accumulation for future projects.
\item \textbf{Demographics:} TFR 2.5-3.5. Growing population.
\item \textbf{Examples:} Victorian Britain, USA 1800-1900, Meiji Japan, Israel
\end{itemize}

\textbf{+1.0 (Pure Metamorphosis):}
\begin{itemize}
\item \textbf{Rhetoric:} ``Infinite growth,'' ``conquest of nature,'' ``apotheosis.''
\item \textbf{Policy:} No limits accepted. Faustian striving. All risk acceptable.
\item \textbf{Culture:} Total subordination of present to future.
\item \textbf{Economics:} Extreme investment. Present consumption minimized.
\item \textbf{Demographics:} TFR 4.0+. Population explosion encouraged.
\item \textbf{Example:} Early American frontier, Genghis Khan's Mongol Empire
\end{itemize}

\textbf{Key Indicators for T-Axis:}

{\small
\begin{longtable}[]{@{}p{2.8cm}p{4.2cm}p{4.2cm}@{}}
\toprule
\textbf{Indicator} & \textbf{Homeostasis (-)} & \textbf{Metamorphosis (+)} \\
\midrule
\endhead
\bottomrule
\endlastfoot
\textbf{Rhetoric:} Dominant metaphors & Safety, sustainability, care & Conquest, building, transcendence \\
\textbf{Policy:} Attitude to risk & Precautionary principle & Calculated risk encouraged \\
\textbf{Cultural:} Heroes & Healers, protectors, therapists & Builders, warriors, explorers \\
\textbf{Economic:} Time preference & High (live for today) & Low (invest for tomorrow) \\
\textbf{Demographic:} TFR & \textless1.5 (population decline) & \textgreater2.5 (population growth) \\
\textbf{Infrastructure:} Investment & Maintenance \textgreater{} new building & New projects \textgreater{} maintenance \\
\textbf{Space:} Frontier mentality & Settled, inward-looking & Expansionist, outward-looking \\
\end{longtable}
}

\subsection{VI. V (Vitality) Scoring Rubric}\label{vi.-v-axis-scoring-rubric-vitality}

\textbf{Core Question:} How alive and effective is the civilization?

\textbf{Note:} This section provides detailed measurement rubrics for V (Vitality), the Layer 3 Output variable defined in \Cref{app:lexicon}, \Cref{iii.-the-causal-hierarchy-v10}. Vitality is the final dependent variable—the measured health resulting from a polity's axiological configuration (SORT) and dynamics (Ω/Α).

Vitality is a composite metric of three sub-indices. Score each 0-10, then average.

\subsubsection{Fecundity Sub-Index (0-10)}\label{a.-fecundity-sub-index-0-10}

\textbf{Measures:} Demographic health + innovation rate

\textbf{Scoring:}
\begin{itemize}
\item \textbf{0-2:} Population collapse (TFR \textless{} 1.0). No innovation. Demographic death spiral.
\item \textbf{3-4:} Slow decline (TFR 1.0-1.5). Minimal innovation. Stagnant.
\item \textbf{5-6:} Replacement level (TFR 1.8-2.2). Moderate innovation. Stable.
\item \textbf{7-8:} Growth (TFR 2.3-3.0). High innovation. Dynamic.
\item \textbf{9-10:} Explosive growth (TFR \textgreater{} 3.0). Revolutionary innovation. Golden age.
\end{itemize}

\textbf{Data sources:}
\begin{itemize}
\item TFR (Total Fertility Rate): Census data, birth records
\item Innovation: Patents per capita, scientific publications, technological breakthroughs
\end{itemize}

\subsubsection{Productivity Sub-Index (0-10)}\label{b.-productivity-sub-index-0-10}

\textbf{Measures:} Economic output + capital accumulation

\textbf{Scoring:}
\begin{itemize}
\item \textbf{0-2:} Economic collapse. Negative growth. Capital destruction.
\item \textbf{3-4:} Stagnation or decline. Zero growth. No capital accumulation.
\item \textbf{5-6:} Slow growth (1-2\% GDP/capita annually). Moderate investment.
\item \textbf{7-8:} Strong growth (3-5\% GDP/capita annually). High investment.
\item \textbf{9-10:} Explosive growth (\textgreater5\% GDP/capita annually). Massive capital accumulation.
\end{itemize}

\textbf{Data sources:}
\begin{itemize}
\item GDP per capita (PPP-adjusted)
\item Capital stock growth
\item Infrastructure investment as \% of GDP
\item Labor productivity growth
\end{itemize}

\subsubsection{Synergy Sub-Index (0-10)}\label{c.-synergy-sub-index-0-10}

\textbf{Measures:} Social cohesion + institutional effectiveness (proxy for Ω)

\textbf{Scoring:}
\begin{itemize}
\item \textbf{0-2:} Civil war, total state collapse. Zero trust. Ω roughly 0-0.2.
\item \textbf{3-4:} Chronic instability. Low trust. Weak institutions. Ω roughly 0.2-0.4.
\item \textbf{5-6:} Functional but strained. Moderate trust. Decent institutions. Ω roughly 0.5-0.7.
\item \textbf{7-8:} Cohesive society. High trust. Strong institutions. Ω roughly 0.7-0.9.
\item \textbf{9-10:} Near-perfect cohesion. Very high trust. Elite institutions. Ω roughly 0.9-1.0.
\end{itemize}

\textbf{Data sources:}
\begin{itemize}
\item Social trust surveys (World Values Survey, etc.)
\item Civil violence indices
\item Corruption indices (Transparency International)
\item Rule of law indices
\item State capacity metrics
\end{itemize}

\subsubsection{Computing Final V-Score:}\label{computing-final-v-score}

V = (Fecundity + Productivity + Synergy) / 3

\textbf{Rationale for formula:} This v1.0 composite uses equal weighting and arithmetic mean as a working definition. The three sub-indices (F/P/S) were selected to capture distinct dimensions of civilizational health: generative capacity (Fecundity), material output (Productivity), and internal coordination (Synergy). Equal weighting treats them as comparably important; arithmetic mean allows partial compensation (a civilization can survive temporary weakness in one domain if strong in others). Alternative formulations—multiplicative (requiring all three), weighted average (prioritizing one dimension), or additional sub-indices—may prove superior pending empirical validation. This formula is a starting operationalization, not a derived necessity.

\textbf{Example:}
\begin{itemize}
\item Fecundity: 7 (TFR 2.5, high innovation)
\item Productivity: 8 (4\% GDP growth)
\item Synergy: 6 (moderate trust, functional institutions)
\item \textbf{V = (7+8+6)/3 = 7.0}
\end{itemize}

\subsection{VII. Ω and Α Measurement Guidelines}\label{vii.-ux3c9-and-ux3b1-measurement-guidelines}

\subsubsection{Ω (Coherence) Measurement (0 to 1)}\label{ux3c9-coherence-measurement-0-to-1}

\textbf{Definition:} Internal unity. Absence of factional conflict.

\textbf{Scoring (0 = total fragmentation, 1 = perfect unity):}

\textbf{0.0-0.2:} Civil war, state collapse, genocidal violence
\textbf{0.3-0.4:} Chronic low-level civil conflict, coup-prone, failed state
\textbf{0.5-0.6:} Functional but contested, polarized politics, low trust
\textbf{0.7-0.8:} Cohesive with manageable tensions, high trust, stable
\textbf{0.9-1.0:} Near-perfect unity, totalitarian conformity OR genuine organic consensus

\textbf{Indicators:}
\begin{itemize}
\item Civil violence: Coups, riots, assassinations, terrorism
\item Political polarization: Vote splits, party fragmentation
\item Social trust: Survey data, ethnic/religious tensions
\item Elite cohesion: Intra-elite conflict or consensus
\end{itemize}

\subsubsection{Α (Action Vector) Measurement (-1 to +1)}\label{ux3b1-action-vector-measurement--1-to-1}

\textbf{Definition:} Net civilizational output. Entropic (-1) to Syntropic (+1).

\textbf{Scoring:}

\textbf{-1.0 to -0.5 (Highly Entropic):} Civilization actively destroying order
\begin{itemize}
\item Genocides, civil wars, cultural revolution destroying heritage
\item Examples: Khmer Rouge Cambodia, ISIS Caliphate, Late Western Roman Empire
\end{itemize}

\textbf{-0.4 to -0.1 (Mildly Entropic):} Net destruction \textgreater{} creation
\begin{itemize}
\item Declining infrastructure, institutional decay, cultural amnesia
\item Examples: Modern Venezuela, Detroit post-1970, Late Soviet Union
\end{itemize}

\textbf{0.0 (Neutral):} Maintenance mode. Stasis.
\begin{itemize}
\item Infrastructure maintained but not expanded
\item Examples: Sleepy towns, stable low-growth economies
\end{itemize}

\textbf{+0.1 to +0.4 (Mildly Syntropic):} Net creation \textgreater{} destruction
\begin{itemize}
\item Modest infrastructure building, cultural production
\item Examples: Steady-state economies, mature civilizations in homeostasis
\end{itemize}

\textbf{+0.5 to +1.0 (Highly Syntropic):} Civilization rapidly creating order
\begin{itemize}
\item Massive infrastructure projects, golden ages of art/science, rapid expansion
\item Examples: Victorian Britain, USA 1950s-60s, Tang Dynasty China
\end{itemize}

\textbf{Indicators:}
\begin{itemize}
\item Infrastructure: Roads, cities, dams, power grids built vs.~decayed
\item Cultural output: Art, literature, scientific discoveries produced
\item Territory: Geographic expansion or contraction
\item Institutions: New institutions founded vs.~collapsed
\end{itemize}

\subsection{VIII. Data Sources and Methodology}\label{viii.-data-sources-and-methodology}

\textbf{Primary Data Sources:}

\textbf{Tier 1:} Census data, economic statistics, legal codes, archaeological evidence

\textbf{Tier 2:} Historical records, contemporary accounts, linguistic analysis, religious/philosophical texts

\textbf{Tier 3:} Modern scholarship, comparative inference, cultural artifacts

\textbf{Triangulation:} Use multiple sources across tiers; flag contradictions; weight Tier 1 heavily; note Tier 3 reliance.

\subsection{IX. Inter-Rater Reliability and Calibration}\label{ix.-inter-rater-reliability-and-calibration}

\textbf{Mitigation strategies:} Calibration exercises using anchor civilizations; explicit reasoning documentation; blind scoring; consensus protocols for divergent scores (>0.3).

\textbf{Acceptable variance:} Within 0.2 (acceptable), within 0.3 (borderline), >0.3 (requires adjudication).

\textbf{Anchor civilizations:} Classical Sparta (S+0.9, O+0.8, R-0.4, T+0.8), USA 1950 (S-0.5, O-0.6, R+0.5, T+0.6), Late Rome 400 CE (S-0.3, O+0.4, R-0.7, T-0.8), Victorian Britain (S-0.6, O-0.4, R+0.6, T+0.9), Modern Sweden 2020 (S-0.4, O+0.4, R+0.3, T-0.7).

\subsection{X. Limitations and Caveats}\label{x.-limitations-and-caveats}

\textbf{1. Historical Data Sparsity:} Ancient civilizations have limited data; scores become more speculative with historical distance.

\textbf{2. Cultural Bias:} Scorers bring axiological orientations. Mitigation: diverse raters, averaging, bias documentation.

\textbf{3. Temporal Granularity:} Score defined periods (``Roman Republic 133-27 BCE''), not entire civilizations (``Rome'').

\textbf{4. Internal Heterogeneity:} Large empires contain variation. Specify whose axiology you're scoring (elite/median/official ideology).

\textbf{5. Rubric Iteration:} Version 1.0. Expect refinement through empirical validation.

\textbf{6. False Precision:} Real confidence intervals likely ±0.2-0.3 for historical cases.

\subsection{XI. Worked Example: Scoring Imperial Rome (27 BCE - 180 CE)}\label{xi.-worked-example-scoring-imperial-rome-27-bce---180-ce}

\textbf{Period:} Pax Romana (Augustus through Marcus Aurelius, \textasciitilde200 years)

\textbf{Evidence sources:} Roman law codes, citizenship policy, tax records, archaeological evidence, governance structure, literature (Virgil, Tacitus), architecture.

\textbf{SORT Scores:}
\begin{itemize}
\item \textbf{S: +0.3} — Individual citizenship rights, but family (paterfamilias) and tribal identity (Roman vs. barbarian) paramount
\item \textbf{O: +0.4} — Codified law, strong central administration, but local customs incorporated
\item \textbf{R: +0.2} — Empirical engineering (aqueducts, roads), pragmatic military tactics, but augury and religious authority still influential
\item \textbf{T: +0.3} — Shift from expansion to consolidation; building projects continued but Pax Romana prioritized stability
\item \textbf{V: 7.5} — Fecundity 7 (TFR~2.5-3), Productivity 8 (strong GDP growth), Synergy 7.5 (cohesive, high trust)
\end{itemize}

\textbf{Uncertainty:} S-Axis (±0.2), R-Axis (±0.3)

\section{Coda: From Rubrics to Research}\label{coda-from-rubrics-to-research}

These rubrics are tools, not truth. They operationalize the SORT framework for systematic empirical testing. Their validity will be determined by:

\begin{enumerate}
\item   \textbf{Inter-rater reliability:} Do independent scorers converge?
\item   \textbf{Predictive validity:} Do SORT scores predict civilizational outcomes?
\item   \textbf{Falsifiability:} Can we find cases where the rubrics produce absurd results?
\end{enumerate}

Use them. Test them. Break them. Report back.

The work of validation is distributed. This is your starting point.

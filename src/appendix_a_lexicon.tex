\chapter{The Complete Lexicon}
\label{app:lexicon}

This appendix serves as the technical reference for the civilizational physics framework. It provides strict, Gnostic definitions and formulas for all canonical variables, organized by epistemological confidence.

\subsection{Epistemological Tiers}\label{epistemological-tiers}

\textbf{TIER 1 - CORE CANON (High Confidence):}
The SORT axes (S/O/R/T), master variables (Ω/Α), and modifier metrics (V/C) constitute the battle-tested foundation presented in the main text. These have demonstrated robust explanatory power across historical case studies and form the essential framework. Use these with confidence.

\textbf{TIER 2 - RESEARCH EXTENSIONS (Exploratory):}
The State Potentials (Ψ/Ν/Κ) explored in \Cref{app:research} represent promising research directions for predictive mechanics. These are working hypotheses requiring systematic empirical validation, not settled law. See \Cref{app:research} for full technical details, current validation status, and research roadmap.

This tiered approach allows the core framework to remain stable while technical extensions evolve through empirical testing.

\subsection{Notation Conventions}\label{notation-conventions}

This framework uses context-appropriate notation to balance precision with readability. Understanding these conventions will help you navigate the technical sections.

\subsubsection{Archetype Notation (Pure Corner Cases)}\label{archetype-notation-pure-corner-cases}

\textbf{Format:} \textbf{{[}S± O± R± T±{]}}

\textbf{Usage:} Labeling the 16 pure-form archetypes where all axes are at extreme values (-1 or +1)

\textbf{Example:} \textbf{{[}S- O- R+ T+{]} \textbar{} The Astral Libertarian}
\begin{itemize}
\item S- = Individual sovereignty
\item O- = Emergent order
\item R+ = Gnostic epistemology
\item T+ = Metamorphic telos
\end{itemize}

\textbf{Neutral/Zero Notation:} For rare cases where an axis is at zero (balanced/neutral), use middle dot: \textbf{S·}
\begin{itemize}
\item Example: \textbf{{[}S· O+ R· T+{]}} indicates balanced Sovereignty and integrated Reality
\item Alternative if middle dot unavailable: \textbf{S\textasciitilde{}} (tilde)
\end{itemize}

\textbf{Why square brackets?} Distinguishes archetypes from measured states. \textbf{Why letter notation?} Self-documenting without memorizing axis order.

\subsubsection{The 16 Archetypal Configurations}\label{the-16-archetypal-configurations}

The four SORT axes at extreme values ($\pm$1) generate $2^4 = 16$ pure-form archetypes. These represent theoretical corners of the possibility space---most real civilizations occupy intermediate positions, but understanding the extremes reveals the framework's generative logic.

\textbf{Organized by T-Axis:} The most fundamental division is between Metamorphic (T+) Foundries and Homeostatic (T-) Hospices.

\needspace{10\baselineskip}
\textbf{THE EIGHT FOUNDRY ARCHETYPES (T+ = Metamorphic)}

These configurations optimize for growth, transformation, and expansion. All pursue transcendent goals requiring sustained effort and sacrifice.

\begin{itemize}
\item \textbf{\sortarch{-}{-}{-}{+} \textbar{} The Psychedelic Revolutionary:} Seeks spiritual transformation through decentralized, myth-driven exploration. Extremely rare at civilizational scale. \textit{Example: 1960s counterculture movements, psychedelic communes.}

\item \textbf{\sortarch{-}{-}{+}{+} \textbar{} The Astral Libertarian:} Rational individuals coordinate voluntarily toward ambitious technological goals. High-trust, high-competence requirement makes this fragile at scale. \textit{Example: Silicon Valley at its best, early American frontier, cypherpunk vision.}

\item \textbf{\sortarch{-}{+}{-}{+} \textbar{} The Utopian Social Architect:} Top-down visionary attempts to redesign society based on mythic ideal while claiming to serve individuals. Often becomes authoritarian despite S- rhetoric. \textit{Example: Certain techno-utopian movements, failed communes with charismatic leaders.}

\item \textbf{\sortarch{-}{+}{+}{+} \textbar{} The Transhumanist Engineer-King:} Centralized technical authority guides humanity toward post-human destiny using empirical methods. Benevolent dictatorship of engineers. \textit{Example: Certain AI safety visions, enlightened technocracy proposals.}

\item \textbf{\sortarch{+}{-}{-}{+} \textbar{} The Rising Nationalist Tribe:} Organic ethnic/national movement mobilized by shared myth toward historical destiny. Decentralized but unified by common identity. \textit{Example: Early nationalist movements, tribal expansion phases, certain populist risings.}

\item \textbf{\sortarch{+}{-}{+}{+} \textbar{} The Techno-Primitivist Collective:} High-Gnosis collective that rejects large-scale centralized design for organic local coordination. Rare—requires sustained competence without hierarchy. \textit{Example: Theoretical (limited historical instantiation), certain anarcho-syndicalist visions.}

\item \textbf{\sortarch{+}{+}{-}{+} \textbar{} The Totalitarian Superstate:} Centralized revolutionary state mobilized by political myth toward radical transformation. Historically common and highly pathological. \textit{Example: Soviet Union, Maoist China, Khmer Rouge, Nazi Germany.}

\item \textbf{\sortarch{+}{+}{+}{+} \textbar{} The Gnostic Hive-Mind:} Perfectly coordinated collective intelligence pursuing transcendent goal with full empirical rigor. The theoretical optimal Foundry if it can be achieved without crushing individual agency. \textit{Example: Theoretical optimum (no sustained historical instantiation), certain AI alignment visions.}
\end{itemize}

\needspace{10\baselineskip}
\textbf{THE EIGHT HOSPICE ARCHETYPES (T- = Homeostatic)}

These configurations optimize for stability, comfort, preservation, and managed decline. All prioritize present equilibrium over future transformation.

\begin{itemize}
\item \textbf{\sortarch{-}{-}{-}{-} \textbar{} The Decadent Anarchist:} Atomized individuals pursue subjective meaning in comfortable drift. No shared truth, no coordination, no collective goals. Terminal stage of liberal democracy. \textit{Example: Late Roman Republic elements, contemporary Western Europe, aspects of modern America.}

\item \textbf{\sortarch{-}{-}{+}{-} \textbar{} The Libertarian Watchman:} Minimal state maintains property rights and peace while free market handles coordination. Stable but aimless—no collective ambition beyond preservation. \textit{Example: Idealized minimal state, certain historical merchant republics in decline phase.}

\item \textbf{\sortarch{-}{+}{-}{-} \textbar{} The Stagnant Dogmatic Theocracy:} Rigid religious authority enforces orthodoxy to preserve tradition. Design serves preservation, not transformation. Brittleness through mythos rigidity. \textit{Example: Late-stage theocracies, certain medieval states in calcification phase.}

\item \textbf{\sortarch{-}{+}{+}{-} \textbar{} The Managed Garden:} Competent technocratic management provides comfort and stability. The "WALL-E" scenario—benevolent administration of human theme park. \textit{Example: Certain visions of AI-managed humanity, extreme welfare states.}

\item \textbf{\sortarch{+}{-}{-}{-} \textbar{} The Traditional Static Village:} High-cohesion organic community where highest goal is preserving ancestral ways. Resilient but non-adaptive. \textit{Example: Pre-modern village societies, traditional tribal configurations in homeostatic mode.}

\item \textbf{\sortarch{+}{-}{+}{-} \textbar{} The Declining Republic:} Honest, efficient, empirically-grounded collective manages its own comfortable decline. High competence applied to managed contraction. \textit{Example: Late Roman Republic in certain phases, modern Japan in demographic decline.}

\item \textbf{\sortarch{+}{+}{-}{-} \textbar{} The Post-Totalitarian State:} Cynical elite preserves power through myth-based design, but revolutionary energy exhausted. System exists to perpetuate itself, not achieve goals. \textit{Example: Late-stage USSR (Brezhnev era), post-Mao China before reforms, current North Korea.}

\item \textbf{\sortarch{+}{+}{+}{-} \textbar{} The Benevolent Stagnant Hive-Mind:} Perfectly administered, data-driven collective optimized for stability and comfort. High competence prevents Totalitarian Superstate pathologies, but no growth. Crystal that no longer grows. \textit{Example: Singapore approaching this configuration, certain technocratic governance visions.}
\end{itemize}

\textbf{Usage Notes:}
\begin{itemize}
\item These are theoretical extremes—real civilizations rarely occupy pure corners
\item Certain configurations are stable attractors (Totalitarian Superstate recurs), others historically rare (Techno-Primitivist Collective)
\item The framework is generative: you derive these from 4 axes, not memorize 16 types
\item Historical examples are illustrative, not definitive classifications
\item See \Cref{ch:sort-framework} for detailed analysis of four key archetypes
\item See \Cref{app:case-studies} for in-depth historical case studies
\end{itemize}

\subsubsection{Observed State Notation (Measured Values)}\label{observed-state-notation-measured-values}

\textbf{Format:} \textbf{(S:value, O:value, R:value, T:value)}

\textbf{Usage:} Reporting actual SORT scores for real civilizations, typically with precision to tenths

\textbf{Example:} \textbf{(S:+0.7, O:+0.6, R:+0.3, T:+0.8)}
\begin{itemize}
\item Indicates a civilization with strong Collective tendency, moderate Design preference, slight Gnosis lean, strong Metamorphic drive
\end{itemize}

\textbf{Precision:} Use ±0.1 resolution. Finer granularity (e.g., +0.73) suggests false precision unless justified by measurement methodology.

\subsubsection{Prose Description Notation}\label{prose-description-notation}

\textbf{Format:} \textbf{Axis± (Pole Name)}

\textbf{Usage:} Inline narrative descriptions of axiological characteristics

\textbf{Example:} ``The civilization is \textbf{S+ (Collective)}, \textbf{R+ (Gnostic)}, and \textbf{T+ (Metamorphic)}''

\textbf{Variations:} Full form \textbf{S+ (Collective)} when introducing concepts; compact \textbf{S+} when context is clear.

\subsubsection{Force Vector Notation}\label{force-vector-notation}

\textbf{Format:} \textbf{↑Axis} or \textbf{↓Axis} (with optional pole clarification)

\textbf{Usage:} Describing forces pushing civilizations along SORT dimensions

\textbf{Example:} ``Scarcity generates \textbf{↑R (toward Gnosis)} and \textbf{↑T (toward Metamorphosis)} force vectors'' (Note: ↑R already implies toward +1, so ↑R+ is redundant)

\subsubsection{Layer Signature Notation}\label{layer-signature-notation}

\textbf{Format:} \textbf{Axis±value} (comma-separated, compact)

\textbf{Usage:} Summarizing ideal SORT configurations for institutional layers or components

\textbf{Example:} \textbf{S+1.0, O-0.8, R-0.9, T-0.5} (The Mythos-Poetic Heart signature). Use when specifying target configurations in institutional design, not measuring actual states.

\subsubsection{Mathematical Notation}\label{mathematical-notation}

\textbf{Format:} Pure numeric vectors \textbf{(-1, +1, 0, +1)} or algebraic symbols \textbf{s, o, r, t}

\textbf{Usage:} Formulas, calculations, distance metrics in SORT space

\textbf{Example:} $\sigma_A = \sqrt{[(s_1-s_2)^2 + (o_1-o_2)^2 + (r_1-r_2)^2 + (t_1-t_2)^2]}$

\subsubsection{Axis Order Standard}\label{axis-order-standard}

\textbf{Always S, O, R, T} (Sovereignty, Organization, Reality, Telos) across all notation types.

\subsubsection{Pole Reference Table}\label{pole-reference-table}

\begin{longtable}[]{@{}p{0.8cm}p{3.5cm}p{3.5cm}p{3.5cm}@{}}
\toprule
\textbf{Axis} & \textbf{Negative Pole (-1)} & \textbf{Positive Pole (+1)} & \textbf{Neutral (0)} \\
\midrule
\endhead
\bottomrule
\endlastfoot
\textbf{S} & Individual & Collective & Balanced \\
\textbf{O} & Emergence & Design & Mixed/Pragmatic \\
\textbf{R} & Mythos & Gnosis & Integrated \\
\textbf{T} & Homeostasis & Metamorphosis & Maintenance \\
\end{longtable}

\textbf{Note:} Zero values indicate genuine balance or synthesis, not mere averaging. A polity at R=0 may have \textbf{Integrity} (Gnostic pursuit of better Mythos) rather than confused middle-ground.

\subsubsection{Modifier Axes (V, C)}\label{modifier-axes-v-c}

\textbf{V (Vitality):} Always expressed as 0-10 scale (not -1 to +1)
\begin{itemize}
\item Format: \textbf{V:7.2} or \textbf{V=7.2}
\item Composite of Fecundity, Productivity, Synergy sub-indices
\end{itemize}

\textbf{C (Constraint):} Expressed as -1 to +1 like SORT axes
\begin{itemize}
\item Format: \textbf{C:+0.8} (hegemon) or \textbf{C:-0.6} (vassal)
\item Measures sovereign agency in geopolitical context
\end{itemize}

\subsubsection{Scale-Specific Notation (tSORT, pSORT, bio-SORT)}\label{scale-specific-notation-tsort-psort-bio-sort}

The SORT framework applies holographically across scales. Different prefixes clarify the level of analysis:

\textbf{tSORT (Tribal/Factional SORT):}
\begin{itemize}
\item \textbf{Usage:} Sub-civilizational units---tribes, political movements, factions, social classes
\item \textbf{Format:} Same as civilizational SORT: \textbf{(S:+0.9, O:+0.9, R:-1.0, T:+0.9)}
\item \textbf{Example:} tSORT of Progressive Clergy faction within American polity (\Cref{ch:american-chimera})
\item \textbf{Why ``t''?} Tribal---emphasizes sub-polity in-group dynamics
\end{itemize}

\textbf{pSORT (Personal/Individual SORT):}
\begin{itemize}
\item \textbf{Usage:} Individual human consciousness and personality structure (\Cref{part:integrated-human})
\item \textbf{Format:} Same coordinate system: \textbf{(S:-0.8, O:-0.7, R:+0.9, T:+0.8)}
\item \textbf{Distinction from tSORT:} pSORT is ONE human, tSORT is a GROUP within a polity
\end{itemize}

\textbf{bio-SORT (Cellular/Biological SORT):}
\begin{itemize}
\item \textbf{Usage:} Cellular systems, tissues, organs (\Cref{app:medicine})
\item \textbf{Format:} Same coordinate system: \textbf{(S:+0.9, O:+0.8, R:+0.9, T:-0.9)} (healthy liver)
\end{itemize}

\textbf{Scale hierarchy:} SORT (civilizational) → tSORT (tribal/factional) → pSORT (personal) → bio-SORT (cellular). All scales use identical axis definitions—the \textbf{holographic principle}.

\subsection{I. The Input Vectors (Layer 1: The Axiological State)}\label{i.-the-input-vectors-layer-1}

These are the five fundamental, independent axes used to plot a polity's axiological configuration. The sixth variable (V - Vitality) is presented in Section III as a Layer 3 Output.

\subsubsection{1. S - Sovereignty}\label{s---sovereignty}

\begin{itemize}
\item   \textbf{Core Question:} Who is the fundamental unit of moral and political allegiance?
\item   \textbf{Range:} -1.0 to +1.0
\item   \textbf{-1.0 Pole (Individual):} The sovereign unit is the individual. The polity exists to maximize personal liberty and agency. Archetype: Classical Liberalism.
\item   \textbf{+1.0 Pole (Collective):} The sovereign unit is the group (tribe, nation). The polity exists to maximize group survival, cohesion, and glory. Archetype: Ancient Sparta.
\end{itemize}

\subsubsection{2. O - Organization}\label{o---organization}

\begin{itemize}
\item   \textbf{Core Question:} How is order created and maintained?
\item   \textbf{Range:} -1.0 to +1.0
\item   \textbf{-1.0 Pole (Emergence):} Order is a bottom-up, spontaneous phenomenon that emerges from the free interactions of agents. Archetype: A free market, common law.
\item   \textbf{+1.0 Pole (Design):} Order is a top-down, architected phenomenon that is consciously planned and imposed by a central authority. Archetype: A command economy, a civil law code.
\end{itemize}

\subsubsection{3. R - Reality}\label{r---reality}

\begin{itemize}
\item   \textbf{Core Question:} What is the ultimate source of truth and authority for decision-making?
\item   \textbf{Range:} -1.0 to +1.0
\item   \textbf{-1.0 Pole (Mythos):} Truth is found in narrative, tradition, religion, and social consensus. It is holistic, contextual, and often unfalsifiable. It provides meaning.
\item   \textbf{+1.0 Pole (Gnosis):} Truth is found in empirical data, logical deduction, and falsifiable experimentation. It is analytical, deconstructed, and impersonal. It provides accuracy.
\end{itemize}

\subsubsection{4. T - Telos}\label{t---telos}

\begin{itemize}
\item   \textbf{Core Question:} What is the ultimate purpose or grand strategy of the polity?
\item   \textbf{Range:} -1.0 to +1.0
\item   \textbf{-1.0 Pole (Homeostasis):} The purpose is Preservation. The goal is to maintain stability, reduce risk, and preserve the current state of being. Archetype: The Hospice.
\item   \textbf{+1.0 Pole (Metamorphosis):} The purpose is Transformation. The goal is to strive, grow, build, and transcend the current state of being. Archetype: The Foundry.
\end{itemize}

\subsubsection{5. C - Constraint}\label{c---constraint}

\begin{itemize}
\item   \textbf{Core Question:} How free is the polity to act without external interference?
\item   \textbf{Range:} -1.0 to +1.0
\item   \textbf{-1.0 Pole (Constrained / Vassal):} The polity's actions are determined or heavily constrained by an external, hegemonic power. Archetype: Vichy France.
\item   \textbf{+1.0 Pole (Unconstrained / Hegemon):} The polity's actions are unconstrained. It sets the rules of the geopolitical game for others. Archetype: The Roman Empire at its peak.
\end{itemize}

\subsection{II. The Master Dynamics (Layer 2: The Ground Truth)}\label{ii.-the-master-dynamics}

These are the two master variables that compress SORT complexity into actionable diagnostics. Unlike the input axes, these are \textbf{measured outputs}—not axiological settings but empirical observations of system behavior.

\subsubsection{1. Ω (Omega) - State Coherence (The Unity Variable)}\label{omega-state-coherence}

\begin{itemize}
\item   \textbf{Core Question:} How unified or internally conflicted is the polity in reality?
\item   \textbf{Range:} 0.0 (Total Incoherence) to 1.0 (Perfect Coherence)
\item   \textbf{Canonical Formula (v1.0):}

  $\Omega = 1 - \sigma_A$

  (where $\sigma_A$ is the empirically measured Axiological Volatility between tribes)
\item   \textbf{Measurement Method:} For simple polities: direct calculation via SORT distance formula (line 149). For complex multi-tribal polities: power-weighted variance calculation detailed in \Cref{app:case-studies}. For v1.0 practical scoring: qualitative assessment via indicators in \Cref{app:scoring-rubrics}, \Cref{ux3c9-coherence-measurement-0-to-1}.
\item   \textbf{Gnostic Deconstruction:} Ω is a \textbf{direct, empirical measurement} of the polity's internal friction and Synergy.
\item   \textbf{Interpretation:} Ω is the efficiency of the engine. A high-Ω polity can effectively transmit its energy into action. A low-Ω polity wastes its energy on internal conflict (the Iron Law of Coherence).
\end{itemize}

\subsubsection{2. Α (Alpha) - Action Vector (The Output Variable)}\label{alpha-action-vector}

\begin{itemize}
\item   \textbf{Core Question:} What is the polity's actual, observed, net effect on the world?
\item   \textbf{Range:} -1.0 (Maximally Entropic) to +1.0 (Maximally Syntropic)
\item   \textbf{Canonical Derivation:} Empirical Assessment. This is a POSIWID-based measurement, not a formulaic derivation.
\item   \textbf{Measurement Method:} Infrastructure built or destroyed, territory gained or lost, order created or annihilated, net effect on human flourishing.
\item   \textbf{Gnostic Deconstruction:} Α is the \textbf{final, irrefutable ground truth}. It is the measurement of what the civilization actually \emph{does}, not what it claims or intends.
\item   \textbf{Interpretation:} Positive Α indicates syntropy (order creation). Negative Α indicates entropy (order destruction). Near-zero Α indicates homeostasis or paralysis.
\end{itemize}

\subsection{III. The Causal Hierarchy (V1.0 Core Model)}\label{iii.-the-causal-hierarchy-v10}

The V1.0 framework operates in three clean layers:

\textbf{LAYER 1: THE AXIOLOGICAL STATE (The Inputs)}
\begin{itemize}
\item SORT axes (S, O, R, T): The civilization's value configuration
\item Constraint modifier (C): External geopolitical freedom
\end{itemize}

\textbf{LAYER 2: THE MEASURED DYNAMICS (The Ground Truth)}
\begin{itemize}
\item Ω (Coherence): Calculated from axiological variance ($1 - \sigma_A$)
\item Α (Action Vector): Empirically observed via POSIWID
\end{itemize}

\textbf{LAYER 3: THE FINAL OUTCOME (The Score)}

\subsubsection{V - Vitality (The Final Dependent Variable)}\label{v---vitality-output}

\begin{itemize}
\item   \textbf{Core Question:} How healthy and effective is the polity in reality?
\item   \textbf{Range:} 0.0 (Civilizational Death) to 10.0 (Apotheosis)
\item   \textbf{Definition:} A composite, empirical metric of a polity's demonstrated Aliveness. It is the \textbf{net result} of the system's axiological configuration (SORT) and resulting dynamics (Ω/Α) over time, measured via three sub-indices:

  \begin{itemize}
  \item
    \textbf{Fecundity:} Demographic health (TFR) and the rate of novelty generation (innovation).
  \item
    \textbf{Productivity:} Economic health and capital accumulation.
  \item
    \textbf{Synergy:} Social health and trust (a proxy for Ω).
  \end{itemize}
\item \textbf{Causal Role:} V is \textbf{NOT an input} that shapes dynamics. It is the \textbf{outcome} that results from them. The causal arrow flows: \textbf{SORT → Ω/Α → V} for diagnostic purposes. V is "The Score"—the final measure of whether the system's configuration and dynamics are producing Aliveness. \textit{Note: In reality, V forms feedback loops with Ω/Α over time (e.g., demographic health affects future coherence), but V is treated as the dependent variable for analytical clarity.}
\item \textbf{Measurement:} Detailed scoring rubrics for the three sub-indices appear in \Cref{app:scoring-rubrics}, \Cref{vi.-v-axis-scoring-rubric-vitality}.
\end{itemize}

\textbf{Other Layer 3 Outputs:}
\begin{itemize}
\item Trajectory predictions based on phase space position (Four States)
\item Predicted failure modes and timescales
\end{itemize}

This streamlined hierarchy reflects what the framework actually does: it uses SORT to understand Ω and Α in order to diagnose V and predict trajectories.

\textbf{For the V0.1 research program} exploring intermediate processing layers (Telic Potential Ψ, Gnomic Potential Ν, Action Potential Κ), see \Cref{app:research}. These represent promising research directions for predictive mechanics but require systematic empirical validation before integration into the core framework.

\subsection{IV. The Master Reference Table (V1.0 Core)}\label{iv.-the-master-reference-table-v10}

This table provides the complete, canonical summary of all V1.0 variables, symbols, and formulas.

{\small
\begin{longtable}[]{@{}p{0.6cm}p{1.6cm}p{1.8cm}p{0.9cm}p{0.9cm}p{3.2cm}@{}}
\toprule
\textbf{Sym\-bol} & \textbf{Canonical Name} & \textbf{Colloquial Name(s)} & \textbf{Layer} & \textbf{Range} & \textbf{Core Question / Function} \\
\midrule
\endhead
\bottomrule
\endlastfoot
\textbf{S} & SOVEREIGNTY & Identity, Agency/Communion & 1. Input & -1 to +1 & Who are we? (The Self) \\
\textbf{O} & ORGANIZATION & Method, Order/Chaos & 1. Input & -1 to +1 & How do we build? (The World) \\
\textbf{R} & REALITY & Epistemology & 1. Input & -1 to +1 & How do we know? (The World) \\
\textbf{T} & TELOS & Purpose, Future/Present & 1. Input & -1 to +1 & Why are we here? (Time) \\
\textbf{V} & VITALITY & Aliveness, The Score & 3. Output & 0 to 10 & How healthy are we? (Final Outcome) \\
\textbf{C} & CONSTRAINT & Agency, The Playing Field & 1. Input & -1 to +1 & How free are we to act? (Environment) \\
& & & & & \\
\textbf{Ω} & STATE COHERENCE & Synergy, Unity & 2. Dynamics & 0 to 1 & The \textbf{Measured} Internal Friction. (Ω = 1 - $\sigma_A$) \\
\textbf{Α} & ACTION VECTOR & Net Effect, Output & 2. Dynamics & -1 to +1 & The \textbf{Measured} External Effect. (Empirical via POSIWID) \\
\end{longtable}
}

\subsection{V. The V1.0 Engine Summarized}\label{v.-the-v10-engine-summarized}

This lexicon is not a list. It is the schematic for a diagnostic engine. The flow of logic is:

\begin{enumerate}
\item   A polity's fundamental \textbf{axiological settings (SORT)}, as shaped by its history and environment (C), determine its configuration in value-space.
\item   We measure the polity's \textbf{internal unity (Ω)} by calculating axiological variance across tribes. High Ω = coherent. Low Ω = fragmented.
\item   We observe the polity's \textbf{actual output (Α)} via POSIWID—what it demonstrably does, not what it claims or intends.
\item   The Ω-Α coordinates place the polity in Phase Space, revealing its state (ALPHA, BETA, GAMMA, ENTROPIC) and predicted trajectory.
\item   The ultimate measure of success is long-term \textbf{Vitality (V)}—sustained Aliveness over time.
\end{enumerate}

This is the complete, falsifiable V1.0 framework for civilizational diagnostics. For speculative extensions exploring predictive mechanics (V0.1 State Potentials), see \Cref{app:research}.

